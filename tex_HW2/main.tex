\documentclass[a4paper,12pt]{article}
\usepackage[a4paper, top = 1cm, bottom = 1.5cm, left = 1cm, right = 1cm, marginparwidth = 1cm]{geometry}

\usepackage{cmap}					% поиск в PDF
\usepackage[T2A]{fontenc}			% кодировка
\usepackage[utf8]{inputenc}			% кодировка исходного текста
\usepackage[english,russian]{babel}	% локализация и переносы

\usepackage{amsmath}
\usepackage{amssymb}
\usepackage{cancel}
\usepackage{graphicx}
\usepackage{float}
\usepackage{subfig}
\usepackage{tabularx}
\usepackage{longtable}
\usepackage[colorlinks=true, citecolor=red, linkcolor=red, urlcolor=blue]{hyperref}

\DeclareMathOperator\arctanh{arctanh}
\numberwithin{equation}{section}
\title{Второе задание}
\author{
	Хохлов Алексей \\
}

\begin{document}
	\maketitle
	

	%1
	\section{Проекция на шар}
	
	Если $\Vert \mathbf{a} \Vert_2 \leqslant 1$, то, очевидно, проекция $\mathbf{p} = \mathbf{a}$.
	
	Если же $\Vert \mathbf{a} \Vert_2 > 1$, очевидно, что искомая точка --- точка пересечения прямой, соединяющей центр сферы и $\mathbf{a}$, с поверхностью единичной сферы. Другими словами, искомая точка $\mathbf{p} = \dfrac{\mathbf{a}}{\Vert \mathbf{a} \Vert_2 }$. Чтобы доказать это, покажем, что $\forall \mathbf{x} \in \mathbf{X}  $ верно $\Vert \mathbf{a} - \mathbf{x} \Vert_2 \geqslant  \Vert \mathbf{a} - \mathbf{p} \Vert_2$.
	
	Для правой стороны неравенства
	
	\begin{equation}
	\Vert \mathbf{a} - \mathbf{p} \Vert_2 = \Vert \mathbf{a} - \frac{\mathbf{a}}{\Vert \mathbf{a} \Vert_2} \Vert_2 =  ( 1 - \frac{1}{\Vert \mathbf{a} \Vert_2}) \Vert \mathbf{a} \Vert_2  = \Vert \mathbf{a} \Vert_2 - 1
	\end{equation}
	
	Для левой стороны неравенства
	
	\begin{equation}
	\Vert \mathbf{a} - \mathbf{x} \Vert_2 \geqslant \Vert \mathbf{a} \Vert_2 - \Vert \mathbf{x} \Vert_2 \geqslant \Vert \mathbf{a} \Vert_2 - 1
	\end{equation}
	
	Видим, что для любой точки в единичном шаре точка $\mathbf{p} = \dfrac{\mathbf{a}}{\Vert \mathbf{a} \Vert_2 }$ является решением задачи
	
	\begin{equation}
	\Vert \mathbf{a} - \mathbf{x} \Vert_2 \rightarrow \text{min}
	\end{equation}
	
	%2
	\section{Минимизация скалярного произведения}
	
	Искомая точка
	
	\begin{align}
	p_i &=
	\left\{
	\begin{aligned}
	1 ,\; & i = k\\
	0 ,\; & i \neq k
	\end{aligned}
	\right. \\
	\end{align}
	
	где $k$ --- такая, что $c_k = \text{min}(c_1, c_2,...,c_n)$. Докажем это.
	
	Поскольку $c_k \leqslant c_i$ для любого $i$, то $c_k x_i \leqslant c_i x_i$. Значит, что
	
	\begin{equation}
	\left\langle \mathbf{c},\mathbf{p} \right\rangle  = c_k = c_k \sum\limits_{i=1}^{n}  x_i =  \sum\limits_{i=1}^{n} c_k x_i \leqslant \sum\limits_{i=1}^{n} c_i x_i = \left\langle \mathbf{c},\mathbf{x} \right\rangle
	\end{equation}
	
	
	%3
	\section{Максимизация функции правдоподобия}
	
	В решении использованы формулы
	
	\begin{equation}
	\label{prove32}
	\frac{\partial}{\partial \mathbf{A}}(\text{tr}(\mathbf{A}^{-1}\mathbf{B})) = -(\mathbf{A}^{-1}\mathbf{B}\mathbf{A}^{-1})^{\text{T}}
	\end{equation}
	
	\begin{equation}
	\label{det31}
	\frac{\partial }{\partial \mathbf{A}} \text{det} (\mathbf{A})= \text{det} (\mathbf{A}) \mathbf{A}^\text{-T}
	\end{equation}
	
	\begin{equation}
	\label{abc}
	\text{tr}(\mathbf{A}\mathbf{B}\mathbf{C}) = \text{tr}(\mathbf{B}\mathbf{C}\mathbf{A})
	\end{equation}
	
	%3.1
	
	
	\subsection{}
	Разложим функцию на слагаемые
	\begin{equation}
	f = \frac{m}{2} \log \text{det} \Sigma + \frac 12 \sum\limits_{i=1}^{m} (\mathbf{x}_i -\boldsymbol{\mu} )^\text{T} \Sigma^{-1} (\mathbf{x}_i -\boldsymbol{\mu} ) = f_1 + f_2
	\end{equation}
	
	Найдем производные слагаемых по $\Sigma$.
	
	Для первого слагаемого воспользуемся~\eqref{det31} 
	
	\begin{equation}
	\frac{\partial}{\partial \Sigma} f_1 = \frac m2 \frac{1}{\text{det}\Sigma} \text{det} (\Sigma) \Sigma^{-\text{T}} = \frac m2 \Sigma^{-\text{T}}
	\end{equation}
	
	Для второго слагаемого воспользуемся~\eqref{abc} и~\eqref{prove32}.
	
	\begin{equation}
	\begin{split}
	&\frac{\partial}{\partial \Sigma}  f_2= \frac 12 \frac{\partial}{\partial \Sigma} \text{tr} (\sum\limits_{i=1}^{m} (\mathbf{x}_i -\boldsymbol{\mu} )^\text{T} \Sigma^{-1} (\mathbf{x}_i -\boldsymbol{\mu} )) = \frac 12 \frac{\partial}{\partial \Sigma} \text{tr} ( \Sigma^{-1}\sum\limits_{i=1}^{m}  (\mathbf{x}_i -\boldsymbol{\mu} ) (\mathbf{x}_i -\boldsymbol{\mu} )^\text{T}) = \\
	& =- \frac 12  \left(  \Sigma^{-1} (\sum\limits_{i=1}^{m}  (\mathbf{x}_i -\boldsymbol{\mu} ) (\mathbf{x}_i -\boldsymbol{\mu} )^\text{T}) \Sigma^{-1}\right) ^\text{T} = - \frac 12  \Sigma^{-T} \left( \sum\limits_{i=1}^{m}  (\mathbf{x}_i -\boldsymbol{\mu} ) (\mathbf{x}_i -\boldsymbol{\mu} )^\text{T}\right)^\text{T}  \Sigma^{-T}
	\end{split}
	\end{equation}
	
	Приравняем производную функции к нулю
	
	\begin{equation}
	\frac m2 \Sigma^{-\text{T}} - \frac 12  \Sigma^{-T} \left( \sum\limits_{i=1}^{m}  (\mathbf{x}_i -\boldsymbol{\mu} ) (\mathbf{x}_i -\boldsymbol{\mu} )^\text{T}\right)^\text{T}  \Sigma^{-T} = 0
	\end{equation}
	
	Умножив на $\Sigma$ и транспонировав, получим
	
	\begin{equation}
	\Sigma = \frac 1m  \sum\limits_{i=1}^{m}  (\mathbf{x}_i -\boldsymbol{\mu} ) (\mathbf{x}_i -\boldsymbol{\mu} )^\text{T}
	\end{equation}
	
	%3.2
	\subsection{}
	
	Докажем формулу~\eqref{prove32}. Во-первых, из $\mathbf{A}\mathbf{A}^{-1} = \mathbf{E}$ следует
	
	\begin{equation}
	\frac{\partial}{\partial a_{pq}} \sum\limits_{j=1}^{n} \tilde{a}_{ij} a_{jk} =  \sum\limits_{j=1}^{n} \frac{\partial \tilde{a}_{ij}}{\partial a_{pq}} a_{jk} + \sum\limits_{j=1}^{n} \tilde{a}_{ij}  \frac{\partial a_{jk}}{\partial a_{pq}} = 0
	\end{equation}
	
	Домножая справа на обратную матрицу, получим
	
	\begin{equation}
	\label{inverse_derivative}
	\frac{\partial \tilde{a}_{ij}}{\partial a_{pq}} = - \sum\limits_{k=1}^{n} \sum\limits_{m=1}^{n} \tilde{a}_{im}  \frac{\partial a_{mk}}{\partial a_{pq}} \tilde{a}_{kj}  = - \tilde{a}_{ip}  \tilde{a}_{qj} 
	\end{equation}
	
	Раскроем теперь производную следа
	
	\begin{equation}
	\frac{\partial}{\partial a_{pq}}(\text{tr}(\mathbf{A}^{-1}\mathbf{B})) = \frac{\partial}{\partial a_{pq}} \sum\limits_{i=1}^{n} \sum\limits_{j=1}^{n} \tilde{a}_{ij} b_{ji} = \sum\limits_{i=1}^{n} \sum\limits_{j=1}^{n} \frac{\partial \tilde{a}_{ij}}{\partial a_{pq}} b_{ji}
	\end{equation}
	
	Подставляя значение производной из~\eqref{inverse_derivative}
	
	\begin{equation}
	\frac{\partial}{\partial a_{pq}}(\text{tr}(\mathbf{A}^{-1}\mathbf{B})) = - \sum\limits_{i=1}^{n} \sum\limits_{j=1}^{n} \tilde{a}_{ip}  \tilde{a}_{qj}  b_{ji} =  - \sum\limits_{i=1}^{n} \sum\limits_{j=1}^{n} \tilde{a}_{pi}^\text{T} b_{ij}^\text{T}  \tilde{a}_{jq}^\text{T} 
	\end{equation}
	
	получив в матричном виде искомую формулу~\eqref{prove32}.
	%4
	\section{}
	
	%4a
	\subsection{}

	%4b
	\subsection{}
	
	%4c
	\subsection{}
	
	%4d
	\subsection{}
	
	
	%5
	\section{}
	
	%5.1
	\subsection{}

	%5.2
	\subsection{}

	%5.3
	\subsection{}
	
	
	%6
	\section{}
	
	
	%7
	\section{}
	
\end{document}

